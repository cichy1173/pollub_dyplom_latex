\section{Analiza porównawcza cech wybranych menedżerów pakietów} \label{roz:opis_srodowisk}

Menedżery pakietów Flatpak oraz Snap .... \ref{tab:porownanie_snap_flatpak}


\vspace{0.5cm}
% \subsection{Porównanie menedżerów pakietów Flatpak i Snap}

\begin{xltabular}{\textwidth}{|X|X|X|}
\caption{Porównanie cech uniwersalnych menedżerów pakietów Flatpak i Snap} \label{tab:porownanie_snap_flatpak} \\
\hline
\multicolumn{1}{|c|}{\textbf{-}} & \multicolumn{1}{c|}{\textbf{Flatpak}} & \multicolumn{1}{c|}{\textbf{Snap}} \\ \hline 
\endfirsthead

\multicolumn{3}{c}%
{Tabela \thetable: Porównanie cech uniwersalnych menedżerów pakietów Flatpak i Snap} \\
\hline \multicolumn{1}{|c|}{\textbf{-}} & \multicolumn{1}{c|}{\textbf{Flatpak}} & \multicolumn{1}{c|}{\textbf{Snap}} \\ \hline 
\endhead

\hline \multicolumn{3}{|r|}{{\textit{Dalsza część tabeli na następnej stronie}}} \\ \hline
\endfoot

\hline
\endlastfoot


\textbf{Lorem Ipsum} & \textbf{Lorem Ipsum}, Lorem Ipsum & \textbf{Lorem Ipsum}Lorem Ipsum \\
\hline
\textbf{Dystrybucja aplikacji z wykorzystaniem zdalnego repozytorium} & Lorem Ipsum & Lorem Ipsum  \\
\hline
\textbf{Automatyczne aktualizacje aplikacji} & \textbf{Nie}. & Domyślnie \textbf{tak}. \\
\hline
\textbf{Wymaganie uprawnień root do zarządzania oprogramowaniem} & \textbf{Nie}. & \textbf{Tak}. \\
\hline
\textbf{Realizacja uprawnień} & Lorem Ipsum & Lorem Ipsum  \\
\hline
\textbf{Obsługa obniżania wersji aplikacji} & \textbf{Tak}. & \textbf{Tak}. \\
\hline
\end{xltabular}


Jak wskazuje tabela \ref{tab:porownanie_snap_flatpak}, ...ponieważ... \enquote{loop}, co sprawia, że...

\vspace{0.5cm}

\begin{lstlisting}[ caption={Plik manifestu potrzebny do zbudowania aplikacji Flatpak}, label={lst:flatpak-manifest-1}]
app-id: eu.cichy1173.tabela
runtime: org.kde.Platform
runtime-version: '6.5'
sdk: org.kde.Sdk
command: tabela
\end{lstlisting}

Listing \ref{lst:flatpak-manifest-1} może zostać wykorzystany do zbudowania aplikacji Flatpak. 